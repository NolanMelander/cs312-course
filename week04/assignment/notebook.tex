
% Default to the notebook output style

    


% Inherit from the specified cell style.




    
\documentclass[11pt]{article}

    
    
    \usepackage[T1]{fontenc}
    % Nicer default font (+ math font) than Computer Modern for most use cases
    \usepackage{mathpazo}

    % Basic figure setup, for now with no caption control since it's done
    % automatically by Pandoc (which extracts ![](path) syntax from Markdown).
    \usepackage{graphicx}
    % We will generate all images so they have a width \maxwidth. This means
    % that they will get their normal width if they fit onto the page, but
    % are scaled down if they would overflow the margins.
    \makeatletter
    \def\maxwidth{\ifdim\Gin@nat@width>\linewidth\linewidth
    \else\Gin@nat@width\fi}
    \makeatother
    \let\Oldincludegraphics\includegraphics
    % Set max figure width to be 80% of text width, for now hardcoded.
    \renewcommand{\includegraphics}[1]{\Oldincludegraphics[width=.8\maxwidth]{#1}}
    % Ensure that by default, figures have no caption (until we provide a
    % proper Figure object with a Caption API and a way to capture that
    % in the conversion process - todo).
    \usepackage{caption}
    \DeclareCaptionLabelFormat{nolabel}{}
    \captionsetup{labelformat=nolabel}

    \usepackage{adjustbox} % Used to constrain images to a maximum size 
    \usepackage{xcolor} % Allow colors to be defined
    \usepackage{enumerate} % Needed for markdown enumerations to work
    \usepackage{geometry} % Used to adjust the document margins
    \usepackage{amsmath} % Equations
    \usepackage{amssymb} % Equations
    \usepackage{textcomp} % defines textquotesingle
    % Hack from http://tex.stackexchange.com/a/47451/13684:
    \AtBeginDocument{%
        \def\PYZsq{\textquotesingle}% Upright quotes in Pygmentized code
    }
    \usepackage{upquote} % Upright quotes for verbatim code
    \usepackage{eurosym} % defines \euro
    \usepackage[mathletters]{ucs} % Extended unicode (utf-8) support
    \usepackage[utf8x]{inputenc} % Allow utf-8 characters in the tex document
    \usepackage{fancyvrb} % verbatim replacement that allows latex
    \usepackage{grffile} % extends the file name processing of package graphics 
                         % to support a larger range 
    % The hyperref package gives us a pdf with properly built
    % internal navigation ('pdf bookmarks' for the table of contents,
    % internal cross-reference links, web links for URLs, etc.)
    \usepackage{hyperref}
    \usepackage{longtable} % longtable support required by pandoc >1.10
    \usepackage{booktabs}  % table support for pandoc > 1.12.2
    \usepackage[inline]{enumitem} % IRkernel/repr support (it uses the enumerate* environment)
    \usepackage[normalem]{ulem} % ulem is needed to support strikethroughs (\sout)
                                % normalem makes italics be italics, not underlines
    

    
    
    % Colors for the hyperref package
    \definecolor{urlcolor}{rgb}{0,.145,.698}
    \definecolor{linkcolor}{rgb}{.71,0.21,0.01}
    \definecolor{citecolor}{rgb}{.12,.54,.11}

    % ANSI colors
    \definecolor{ansi-black}{HTML}{3E424D}
    \definecolor{ansi-black-intense}{HTML}{282C36}
    \definecolor{ansi-red}{HTML}{E75C58}
    \definecolor{ansi-red-intense}{HTML}{B22B31}
    \definecolor{ansi-green}{HTML}{00A250}
    \definecolor{ansi-green-intense}{HTML}{007427}
    \definecolor{ansi-yellow}{HTML}{DDB62B}
    \definecolor{ansi-yellow-intense}{HTML}{B27D12}
    \definecolor{ansi-blue}{HTML}{208FFB}
    \definecolor{ansi-blue-intense}{HTML}{0065CA}
    \definecolor{ansi-magenta}{HTML}{D160C4}
    \definecolor{ansi-magenta-intense}{HTML}{A03196}
    \definecolor{ansi-cyan}{HTML}{60C6C8}
    \definecolor{ansi-cyan-intense}{HTML}{258F8F}
    \definecolor{ansi-white}{HTML}{C5C1B4}
    \definecolor{ansi-white-intense}{HTML}{A1A6B2}

    % commands and environments needed by pandoc snippets
    % extracted from the output of `pandoc -s`
    \providecommand{\tightlist}{%
      \setlength{\itemsep}{0pt}\setlength{\parskip}{0pt}}
    \DefineVerbatimEnvironment{Highlighting}{Verbatim}{commandchars=\\\{\}}
    % Add ',fontsize=\small' for more characters per line
    \newenvironment{Shaded}{}{}
    \newcommand{\KeywordTok}[1]{\textcolor[rgb]{0.00,0.44,0.13}{\textbf{{#1}}}}
    \newcommand{\DataTypeTok}[1]{\textcolor[rgb]{0.56,0.13,0.00}{{#1}}}
    \newcommand{\DecValTok}[1]{\textcolor[rgb]{0.25,0.63,0.44}{{#1}}}
    \newcommand{\BaseNTok}[1]{\textcolor[rgb]{0.25,0.63,0.44}{{#1}}}
    \newcommand{\FloatTok}[1]{\textcolor[rgb]{0.25,0.63,0.44}{{#1}}}
    \newcommand{\CharTok}[1]{\textcolor[rgb]{0.25,0.44,0.63}{{#1}}}
    \newcommand{\StringTok}[1]{\textcolor[rgb]{0.25,0.44,0.63}{{#1}}}
    \newcommand{\CommentTok}[1]{\textcolor[rgb]{0.38,0.63,0.69}{\textit{{#1}}}}
    \newcommand{\OtherTok}[1]{\textcolor[rgb]{0.00,0.44,0.13}{{#1}}}
    \newcommand{\AlertTok}[1]{\textcolor[rgb]{1.00,0.00,0.00}{\textbf{{#1}}}}
    \newcommand{\FunctionTok}[1]{\textcolor[rgb]{0.02,0.16,0.49}{{#1}}}
    \newcommand{\RegionMarkerTok}[1]{{#1}}
    \newcommand{\ErrorTok}[1]{\textcolor[rgb]{1.00,0.00,0.00}{\textbf{{#1}}}}
    \newcommand{\NormalTok}[1]{{#1}}
    
    % Additional commands for more recent versions of Pandoc
    \newcommand{\ConstantTok}[1]{\textcolor[rgb]{0.53,0.00,0.00}{{#1}}}
    \newcommand{\SpecialCharTok}[1]{\textcolor[rgb]{0.25,0.44,0.63}{{#1}}}
    \newcommand{\VerbatimStringTok}[1]{\textcolor[rgb]{0.25,0.44,0.63}{{#1}}}
    \newcommand{\SpecialStringTok}[1]{\textcolor[rgb]{0.73,0.40,0.53}{{#1}}}
    \newcommand{\ImportTok}[1]{{#1}}
    \newcommand{\DocumentationTok}[1]{\textcolor[rgb]{0.73,0.13,0.13}{\textit{{#1}}}}
    \newcommand{\AnnotationTok}[1]{\textcolor[rgb]{0.38,0.63,0.69}{\textbf{\textit{{#1}}}}}
    \newcommand{\CommentVarTok}[1]{\textcolor[rgb]{0.38,0.63,0.69}{\textbf{\textit{{#1}}}}}
    \newcommand{\VariableTok}[1]{\textcolor[rgb]{0.10,0.09,0.49}{{#1}}}
    \newcommand{\ControlFlowTok}[1]{\textcolor[rgb]{0.00,0.44,0.13}{\textbf{{#1}}}}
    \newcommand{\OperatorTok}[1]{\textcolor[rgb]{0.40,0.40,0.40}{{#1}}}
    \newcommand{\BuiltInTok}[1]{{#1}}
    \newcommand{\ExtensionTok}[1]{{#1}}
    \newcommand{\PreprocessorTok}[1]{\textcolor[rgb]{0.74,0.48,0.00}{{#1}}}
    \newcommand{\AttributeTok}[1]{\textcolor[rgb]{0.49,0.56,0.16}{{#1}}}
    \newcommand{\InformationTok}[1]{\textcolor[rgb]{0.38,0.63,0.69}{\textbf{\textit{{#1}}}}}
    \newcommand{\WarningTok}[1]{\textcolor[rgb]{0.38,0.63,0.69}{\textbf{\textit{{#1}}}}}
    
    
    % Define a nice break command that doesn't care if a line doesn't already
    % exist.
    \def\br{\hspace*{\fill} \\* }
    % Math Jax compatability definitions
    \def\gt{>}
    \def\lt{<}
    % Document parameters
    \title{prove}
    
    
    

    % Pygments definitions
    
\makeatletter
\def\PY@reset{\let\PY@it=\relax \let\PY@bf=\relax%
    \let\PY@ul=\relax \let\PY@tc=\relax%
    \let\PY@bc=\relax \let\PY@ff=\relax}
\def\PY@tok#1{\csname PY@tok@#1\endcsname}
\def\PY@toks#1+{\ifx\relax#1\empty\else%
    \PY@tok{#1}\expandafter\PY@toks\fi}
\def\PY@do#1{\PY@bc{\PY@tc{\PY@ul{%
    \PY@it{\PY@bf{\PY@ff{#1}}}}}}}
\def\PY#1#2{\PY@reset\PY@toks#1+\relax+\PY@do{#2}}

\expandafter\def\csname PY@tok@w\endcsname{\def\PY@tc##1{\textcolor[rgb]{0.73,0.73,0.73}{##1}}}
\expandafter\def\csname PY@tok@c\endcsname{\let\PY@it=\textit\def\PY@tc##1{\textcolor[rgb]{0.25,0.50,0.50}{##1}}}
\expandafter\def\csname PY@tok@cp\endcsname{\def\PY@tc##1{\textcolor[rgb]{0.74,0.48,0.00}{##1}}}
\expandafter\def\csname PY@tok@k\endcsname{\let\PY@bf=\textbf\def\PY@tc##1{\textcolor[rgb]{0.00,0.50,0.00}{##1}}}
\expandafter\def\csname PY@tok@kp\endcsname{\def\PY@tc##1{\textcolor[rgb]{0.00,0.50,0.00}{##1}}}
\expandafter\def\csname PY@tok@kt\endcsname{\def\PY@tc##1{\textcolor[rgb]{0.69,0.00,0.25}{##1}}}
\expandafter\def\csname PY@tok@o\endcsname{\def\PY@tc##1{\textcolor[rgb]{0.40,0.40,0.40}{##1}}}
\expandafter\def\csname PY@tok@ow\endcsname{\let\PY@bf=\textbf\def\PY@tc##1{\textcolor[rgb]{0.67,0.13,1.00}{##1}}}
\expandafter\def\csname PY@tok@nb\endcsname{\def\PY@tc##1{\textcolor[rgb]{0.00,0.50,0.00}{##1}}}
\expandafter\def\csname PY@tok@nf\endcsname{\def\PY@tc##1{\textcolor[rgb]{0.00,0.00,1.00}{##1}}}
\expandafter\def\csname PY@tok@nc\endcsname{\let\PY@bf=\textbf\def\PY@tc##1{\textcolor[rgb]{0.00,0.00,1.00}{##1}}}
\expandafter\def\csname PY@tok@nn\endcsname{\let\PY@bf=\textbf\def\PY@tc##1{\textcolor[rgb]{0.00,0.00,1.00}{##1}}}
\expandafter\def\csname PY@tok@ne\endcsname{\let\PY@bf=\textbf\def\PY@tc##1{\textcolor[rgb]{0.82,0.25,0.23}{##1}}}
\expandafter\def\csname PY@tok@nv\endcsname{\def\PY@tc##1{\textcolor[rgb]{0.10,0.09,0.49}{##1}}}
\expandafter\def\csname PY@tok@no\endcsname{\def\PY@tc##1{\textcolor[rgb]{0.53,0.00,0.00}{##1}}}
\expandafter\def\csname PY@tok@nl\endcsname{\def\PY@tc##1{\textcolor[rgb]{0.63,0.63,0.00}{##1}}}
\expandafter\def\csname PY@tok@ni\endcsname{\let\PY@bf=\textbf\def\PY@tc##1{\textcolor[rgb]{0.60,0.60,0.60}{##1}}}
\expandafter\def\csname PY@tok@na\endcsname{\def\PY@tc##1{\textcolor[rgb]{0.49,0.56,0.16}{##1}}}
\expandafter\def\csname PY@tok@nt\endcsname{\let\PY@bf=\textbf\def\PY@tc##1{\textcolor[rgb]{0.00,0.50,0.00}{##1}}}
\expandafter\def\csname PY@tok@nd\endcsname{\def\PY@tc##1{\textcolor[rgb]{0.67,0.13,1.00}{##1}}}
\expandafter\def\csname PY@tok@s\endcsname{\def\PY@tc##1{\textcolor[rgb]{0.73,0.13,0.13}{##1}}}
\expandafter\def\csname PY@tok@sd\endcsname{\let\PY@it=\textit\def\PY@tc##1{\textcolor[rgb]{0.73,0.13,0.13}{##1}}}
\expandafter\def\csname PY@tok@si\endcsname{\let\PY@bf=\textbf\def\PY@tc##1{\textcolor[rgb]{0.73,0.40,0.53}{##1}}}
\expandafter\def\csname PY@tok@se\endcsname{\let\PY@bf=\textbf\def\PY@tc##1{\textcolor[rgb]{0.73,0.40,0.13}{##1}}}
\expandafter\def\csname PY@tok@sr\endcsname{\def\PY@tc##1{\textcolor[rgb]{0.73,0.40,0.53}{##1}}}
\expandafter\def\csname PY@tok@ss\endcsname{\def\PY@tc##1{\textcolor[rgb]{0.10,0.09,0.49}{##1}}}
\expandafter\def\csname PY@tok@sx\endcsname{\def\PY@tc##1{\textcolor[rgb]{0.00,0.50,0.00}{##1}}}
\expandafter\def\csname PY@tok@m\endcsname{\def\PY@tc##1{\textcolor[rgb]{0.40,0.40,0.40}{##1}}}
\expandafter\def\csname PY@tok@gh\endcsname{\let\PY@bf=\textbf\def\PY@tc##1{\textcolor[rgb]{0.00,0.00,0.50}{##1}}}
\expandafter\def\csname PY@tok@gu\endcsname{\let\PY@bf=\textbf\def\PY@tc##1{\textcolor[rgb]{0.50,0.00,0.50}{##1}}}
\expandafter\def\csname PY@tok@gd\endcsname{\def\PY@tc##1{\textcolor[rgb]{0.63,0.00,0.00}{##1}}}
\expandafter\def\csname PY@tok@gi\endcsname{\def\PY@tc##1{\textcolor[rgb]{0.00,0.63,0.00}{##1}}}
\expandafter\def\csname PY@tok@gr\endcsname{\def\PY@tc##1{\textcolor[rgb]{1.00,0.00,0.00}{##1}}}
\expandafter\def\csname PY@tok@ge\endcsname{\let\PY@it=\textit}
\expandafter\def\csname PY@tok@gs\endcsname{\let\PY@bf=\textbf}
\expandafter\def\csname PY@tok@gp\endcsname{\let\PY@bf=\textbf\def\PY@tc##1{\textcolor[rgb]{0.00,0.00,0.50}{##1}}}
\expandafter\def\csname PY@tok@go\endcsname{\def\PY@tc##1{\textcolor[rgb]{0.53,0.53,0.53}{##1}}}
\expandafter\def\csname PY@tok@gt\endcsname{\def\PY@tc##1{\textcolor[rgb]{0.00,0.27,0.87}{##1}}}
\expandafter\def\csname PY@tok@err\endcsname{\def\PY@bc##1{\setlength{\fboxsep}{0pt}\fcolorbox[rgb]{1.00,0.00,0.00}{1,1,1}{\strut ##1}}}
\expandafter\def\csname PY@tok@kc\endcsname{\let\PY@bf=\textbf\def\PY@tc##1{\textcolor[rgb]{0.00,0.50,0.00}{##1}}}
\expandafter\def\csname PY@tok@kd\endcsname{\let\PY@bf=\textbf\def\PY@tc##1{\textcolor[rgb]{0.00,0.50,0.00}{##1}}}
\expandafter\def\csname PY@tok@kn\endcsname{\let\PY@bf=\textbf\def\PY@tc##1{\textcolor[rgb]{0.00,0.50,0.00}{##1}}}
\expandafter\def\csname PY@tok@kr\endcsname{\let\PY@bf=\textbf\def\PY@tc##1{\textcolor[rgb]{0.00,0.50,0.00}{##1}}}
\expandafter\def\csname PY@tok@bp\endcsname{\def\PY@tc##1{\textcolor[rgb]{0.00,0.50,0.00}{##1}}}
\expandafter\def\csname PY@tok@fm\endcsname{\def\PY@tc##1{\textcolor[rgb]{0.00,0.00,1.00}{##1}}}
\expandafter\def\csname PY@tok@vc\endcsname{\def\PY@tc##1{\textcolor[rgb]{0.10,0.09,0.49}{##1}}}
\expandafter\def\csname PY@tok@vg\endcsname{\def\PY@tc##1{\textcolor[rgb]{0.10,0.09,0.49}{##1}}}
\expandafter\def\csname PY@tok@vi\endcsname{\def\PY@tc##1{\textcolor[rgb]{0.10,0.09,0.49}{##1}}}
\expandafter\def\csname PY@tok@vm\endcsname{\def\PY@tc##1{\textcolor[rgb]{0.10,0.09,0.49}{##1}}}
\expandafter\def\csname PY@tok@sa\endcsname{\def\PY@tc##1{\textcolor[rgb]{0.73,0.13,0.13}{##1}}}
\expandafter\def\csname PY@tok@sb\endcsname{\def\PY@tc##1{\textcolor[rgb]{0.73,0.13,0.13}{##1}}}
\expandafter\def\csname PY@tok@sc\endcsname{\def\PY@tc##1{\textcolor[rgb]{0.73,0.13,0.13}{##1}}}
\expandafter\def\csname PY@tok@dl\endcsname{\def\PY@tc##1{\textcolor[rgb]{0.73,0.13,0.13}{##1}}}
\expandafter\def\csname PY@tok@s2\endcsname{\def\PY@tc##1{\textcolor[rgb]{0.73,0.13,0.13}{##1}}}
\expandafter\def\csname PY@tok@sh\endcsname{\def\PY@tc##1{\textcolor[rgb]{0.73,0.13,0.13}{##1}}}
\expandafter\def\csname PY@tok@s1\endcsname{\def\PY@tc##1{\textcolor[rgb]{0.73,0.13,0.13}{##1}}}
\expandafter\def\csname PY@tok@mb\endcsname{\def\PY@tc##1{\textcolor[rgb]{0.40,0.40,0.40}{##1}}}
\expandafter\def\csname PY@tok@mf\endcsname{\def\PY@tc##1{\textcolor[rgb]{0.40,0.40,0.40}{##1}}}
\expandafter\def\csname PY@tok@mh\endcsname{\def\PY@tc##1{\textcolor[rgb]{0.40,0.40,0.40}{##1}}}
\expandafter\def\csname PY@tok@mi\endcsname{\def\PY@tc##1{\textcolor[rgb]{0.40,0.40,0.40}{##1}}}
\expandafter\def\csname PY@tok@il\endcsname{\def\PY@tc##1{\textcolor[rgb]{0.40,0.40,0.40}{##1}}}
\expandafter\def\csname PY@tok@mo\endcsname{\def\PY@tc##1{\textcolor[rgb]{0.40,0.40,0.40}{##1}}}
\expandafter\def\csname PY@tok@ch\endcsname{\let\PY@it=\textit\def\PY@tc##1{\textcolor[rgb]{0.25,0.50,0.50}{##1}}}
\expandafter\def\csname PY@tok@cm\endcsname{\let\PY@it=\textit\def\PY@tc##1{\textcolor[rgb]{0.25,0.50,0.50}{##1}}}
\expandafter\def\csname PY@tok@cpf\endcsname{\let\PY@it=\textit\def\PY@tc##1{\textcolor[rgb]{0.25,0.50,0.50}{##1}}}
\expandafter\def\csname PY@tok@c1\endcsname{\let\PY@it=\textit\def\PY@tc##1{\textcolor[rgb]{0.25,0.50,0.50}{##1}}}
\expandafter\def\csname PY@tok@cs\endcsname{\let\PY@it=\textit\def\PY@tc##1{\textcolor[rgb]{0.25,0.50,0.50}{##1}}}

\def\PYZbs{\char`\\}
\def\PYZus{\char`\_}
\def\PYZob{\char`\{}
\def\PYZcb{\char`\}}
\def\PYZca{\char`\^}
\def\PYZam{\char`\&}
\def\PYZlt{\char`\<}
\def\PYZgt{\char`\>}
\def\PYZsh{\char`\#}
\def\PYZpc{\char`\%}
\def\PYZdl{\char`\$}
\def\PYZhy{\char`\-}
\def\PYZsq{\char`\'}
\def\PYZdq{\char`\"}
\def\PYZti{\char`\~}
% for compatibility with earlier versions
\def\PYZat{@}
\def\PYZlb{[}
\def\PYZrb{]}
\makeatother


    % Exact colors from NB
    \definecolor{incolor}{rgb}{0.0, 0.0, 0.5}
    \definecolor{outcolor}{rgb}{0.545, 0.0, 0.0}



    
    % Prevent overflowing lines due to hard-to-break entities
    \sloppy 
    % Setup hyperref package
    \hypersetup{
      breaklinks=true,  % so long urls are correctly broken across lines
      colorlinks=true,
      urlcolor=urlcolor,
      linkcolor=linkcolor,
      citecolor=citecolor,
      }
    % Slightly bigger margins than the latex defaults
    
    \geometry{verbose,tmargin=1in,bmargin=1in,lmargin=1in,rmargin=1in}
    
    

    \begin{document}
    
    
    \maketitle
    
    

    
    \includegraphics{../../images/cs312.png} ***

\section{04 Prove - Assignment}\label{prove---assignment}

\subsection{1 - Objectives}\label{objectives}

\begin{itemize}
\tightlist
\item
  In this assignment, you will edit short videos by applying image
  filters on them.
\end{itemize}

\subsubsection{1.1 - Reading References}\label{reading-references}

\begin{itemize}
\tightlist
\item
  https://docs.opencv.org/3.0-beta/modules/imgproc/doc/filtering.html
\item
  https://docs.opencv.org/2.4/modules/imgproc/doc/filtering.html?highlight=filter2d\#filter2d
\item
  https://docs.opencv.org/3.0-beta/doc/py\_tutorials/py\_gui/py\_video\_display/py\_video\_display.html
\end{itemize}

    \subsection{2 - Introduction}\label{introduction}

\begin{itemize}
\tightlist
\item
  In the past assignments, you have been applying filters to single
  images. A video is just a sequence of images.
\item
  The code given in this assignment will allow you to seperate the
  images of a video and to combine them together into a new video file.
\item
  Special care must be taken with the size of the videos (ie., large
  videos will take a long time to process). Also, make sure you
  understand the size (width and height) of the video images and their
  channels. This will be important while applying filters on them.
\end{itemize}

    \begin{Verbatim}[commandchars=\\\{\}]
{\color{incolor}In [{\color{incolor}1}]:} \PY{c+c1}{\PYZsh{} These are the libraries used in this notebook \PYZhy{} you must run this cell}
        \PY{o}{\PYZpc{}}\PY{k}{matplotlib} inline
        \PY{k+kn}{import} \PY{n+nn}{numpy} \PY{k}{as} \PY{n+nn}{np}
        \PY{k+kn}{import} \PY{n+nn}{matplotlib}\PY{n+nn}{.}\PY{n+nn}{pyplot} \PY{k}{as} \PY{n+nn}{plt}
        \PY{k+kn}{import} \PY{n+nn}{cv2}
\end{Verbatim}


    \begin{center}\rule{0.5\linewidth}{\linethickness}\end{center}

\subsection{3 - Assignment}\label{assignment}

This assignment has two parts.

    \subsubsection{3.1 - Part 1 - Add a filter to the given video
file}\label{part-1---add-a-filter-to-the-given-video-file}

\begin{enumerate}
\def\labelenumi{\arabic{enumi}.}
\tightlist
\item
  After you download all of the files for this assignment, you will find
  the file "sample.mp4".
\item
  You will be required to apply a filter to this video - one frame at a
  time and then create a new video from those frames.
\item
  You are free to use any OpenCV/numpy functions or functions from
  previous assignments.
\item
  \textbf{Note} that the goal is for you to show that you were able to
  apply a filter to the video. I need to be able to \textbf{see the
  results} of that filter.
\item
  You will be required to provide a link to your resulting video so that
  I can view it. Use DropBox, OneDrive, YouTube, etc...
\end{enumerate}

    \begin{Verbatim}[commandchars=\\\{\}]
{\color{incolor}In [{\color{incolor}2}]:} \PY{c+c1}{\PYZsh{} Open the sample video}
        \PY{n}{cap} \PY{o}{=} \PY{n}{cv2}\PY{o}{.}\PY{n}{VideoCapture}\PY{p}{(}\PY{l+s+s1}{\PYZsq{}}\PY{l+s+s1}{sample.mp4}\PY{l+s+s1}{\PYZsq{}}\PY{p}{)}
        
        \PY{c+c1}{\PYZsh{} Read the first frame and get the height, width and number of color channels}
        \PY{n}{ret}\PY{p}{,} \PY{n}{frame} \PY{o}{=} \PY{n}{cap}\PY{o}{.}\PY{n}{read}\PY{p}{(}\PY{p}{)}
        \PY{n}{height}\PY{p}{,} \PY{n}{width}\PY{p}{,} \PY{n}{channels} \PY{o}{=} \PY{n}{frame}\PY{o}{.}\PY{n}{shape}
        
        \PY{c+c1}{\PYZsh{} Output video must be the same size as the input in this example}
        \PY{n}{fourcc} \PY{o}{=} \PY{n}{cv2}\PY{o}{.}\PY{n}{VideoWriter\PYZus{}fourcc}\PY{p}{(}\PY{o}{*}\PY{l+s+s1}{\PYZsq{}}\PY{l+s+s1}{mp4v}\PY{l+s+s1}{\PYZsq{}}\PY{p}{)} \PY{c+c1}{\PYZsh{} Be sure to use lower case}
        \PY{n}{out} \PY{o}{=} \PY{n}{cv2}\PY{o}{.}\PY{n}{VideoWriter}\PY{p}{(}\PY{l+s+s1}{\PYZsq{}}\PY{l+s+s1}{output.mp4}\PY{l+s+s1}{\PYZsq{}}\PY{p}{,} \PY{n}{fourcc}\PY{p}{,} \PY{l+m+mf}{25.0}\PY{p}{,} \PY{p}{(}\PY{n}{width}\PY{p}{,} \PY{n}{height}\PY{p}{)}\PY{p}{)}
        
        \PY{c+c1}{\PYZsh{} Release the video in and open it again}
        \PY{n}{cap}\PY{o}{.}\PY{n}{release}\PY{p}{(}\PY{p}{)}
        \PY{n}{cap} \PY{o}{=} \PY{n}{cv2}\PY{o}{.}\PY{n}{VideoCapture}\PY{p}{(}\PY{l+s+s1}{\PYZsq{}}\PY{l+s+s1}{sample.mp4}\PY{l+s+s1}{\PYZsq{}}\PY{p}{)}
        
        \PY{n}{ret} \PY{o}{=} \PY{k+kc}{True}
        \PY{k}{while} \PY{p}{(}\PY{n}{ret} \PY{o+ow}{and} \PY{n}{cap}\PY{o}{.}\PY{n}{isOpened}\PY{p}{(}\PY{p}{)}\PY{p}{)}\PY{p}{:}
            \PY{n}{ret}\PY{p}{,} \PY{n}{frame} \PY{o}{=} \PY{n}{cap}\PY{o}{.}\PY{n}{read}\PY{p}{(}\PY{p}{)}
            \PY{k}{if} \PY{n}{ret} \PY{o}{==} \PY{k+kc}{True}\PY{p}{:}
        
                \PY{c+c1}{\PYZsh{} TODO \PYZhy{} add you filter here}
                \PY{n}{kernel} \PY{o}{=} \PY{n}{np}\PY{o}{.}\PY{n}{ones}\PY{p}{(}\PY{p}{(}\PY{l+m+mi}{5}\PY{p}{,}\PY{l+m+mi}{5}\PY{p}{)}\PY{p}{,}\PY{n}{np}\PY{o}{.}\PY{n}{float32}\PY{p}{)}\PY{o}{/}\PY{l+m+mi}{25}
                \PY{n}{frame} \PY{o}{=} \PY{n}{cv2}\PY{o}{.}\PY{n}{filter2D}\PY{p}{(}\PY{n}{frame}\PY{p}{,}\PY{o}{\PYZhy{}}\PY{l+m+mi}{1}\PY{p}{,}\PY{n}{kernel}\PY{p}{)}
                \PY{n}{frame} \PY{o}{=} \PY{n}{cv2}\PY{o}{.}\PY{n}{bilateralFilter}\PY{p}{(}\PY{n}{frame}\PY{p}{,} \PY{l+m+mi}{9}\PY{p}{,} \PY{l+m+mi}{75}\PY{p}{,} \PY{l+m+mi}{75}\PY{p}{)}
                \PY{c+c1}{\PYZsh{} Write the changed frame out}
                \PY{n}{out}\PY{o}{.}\PY{n}{write}\PY{p}{(}\PY{n}{frame}\PY{p}{)}
                
        \PY{c+c1}{\PYZsh{} When everything done, release the capture}
        \PY{n}{out}\PY{o}{.}\PY{n}{release}\PY{p}{(}\PY{p}{)}
        \PY{n}{cap}\PY{o}{.}\PY{n}{release}\PY{p}{(}\PY{p}{)}
        
        \PY{n+nb}{print}\PY{p}{(}\PY{l+s+s1}{\PYZsq{}}\PY{l+s+s1}{Finished}\PY{l+s+s1}{\PYZsq{}}\PY{p}{)}
\end{Verbatim}


    \begin{Verbatim}[commandchars=\\\{\}]
Finished

    \end{Verbatim}

    \subsubsection{3.1.1 - Resulting video}\label{resulting-video}

TODO - https://youtu.be/0dLk5G65RgU

TODO - Question: Describe your filter and the results (single sentence
answers will not be accepted) I ended up using two filters. I first
started with the Averaging, which used a kernel to average the "noise"
this causes the video to become slightly blured. I then followed up by
adding a bilateralFilter and attempted to see if I could sharpen up the
video. What we have in the link is the final result. A relatively blury
video.

    \begin{center}\rule{0.5\linewidth}{\linethickness}\end{center}

\subsubsection{3.2 - Part 2 - Your own
videos}\label{part-2---your-own-videos}

\begin{enumerate}
\def\labelenumi{\arabic{enumi}.}
\tightlist
\item
  Using your cell phone or camera, take a video of some interesting. Be
  careful not to take a video that is too long or too large in height
  and width. There is software that can edit videos.
\item
  You will be required to create three videos where you apply a
  different filter to each. You can use three different videos if you
  want. Be creative!
\item
  \textbf{Note} that the goal is for you to show that you were able to
  apply a filter to the video. I need to be able to \textbf{see the
  results} of that filter.
\item
  You will be required to provide link(s) to your resulting video so
  that I can view it. Use DropBox, OneDrive, YouTube, etc...
\item
  You can either provide all of your code in the following cell, or
  provide the code for each type of filter in their own cell. It needs
  to be easy to understand what your code does.
\end{enumerate}

    \begin{Verbatim}[commandchars=\\\{\}]
{\color{incolor}In [{\color{incolor}3}]:} \PY{c+c1}{\PYZsh{} Open the sample video}
        \PY{n}{cap} \PY{o}{=} \PY{n}{cv2}\PY{o}{.}\PY{n}{VideoCapture}\PY{p}{(}\PY{l+s+s1}{\PYZsq{}}\PY{l+s+s1}{bubbles.mp4}\PY{l+s+s1}{\PYZsq{}}\PY{p}{)}
        
        \PY{c+c1}{\PYZsh{} Read the first frame and get the height, width and number of color channels}
        \PY{n}{ret}\PY{p}{,} \PY{n}{frame} \PY{o}{=} \PY{n}{cap}\PY{o}{.}\PY{n}{read}\PY{p}{(}\PY{p}{)}
        \PY{n}{height}\PY{p}{,} \PY{n}{width}\PY{p}{,} \PY{n}{channels} \PY{o}{=} \PY{n}{frame}\PY{o}{.}\PY{n}{shape}
        
        \PY{c+c1}{\PYZsh{} Output video must be the same size as the input in this example}
        \PY{n}{fourcc} \PY{o}{=} \PY{n}{cv2}\PY{o}{.}\PY{n}{VideoWriter\PYZus{}fourcc}\PY{p}{(}\PY{o}{*}\PY{l+s+s1}{\PYZsq{}}\PY{l+s+s1}{mp4v}\PY{l+s+s1}{\PYZsq{}}\PY{p}{)} \PY{c+c1}{\PYZsh{} Be sure to use lower case}
        \PY{n}{out} \PY{o}{=} \PY{n}{cv2}\PY{o}{.}\PY{n}{VideoWriter}\PY{p}{(}\PY{l+s+s1}{\PYZsq{}}\PY{l+s+s1}{myoutput1.mp4}\PY{l+s+s1}{\PYZsq{}}\PY{p}{,} \PY{n}{fourcc}\PY{p}{,} \PY{l+m+mf}{25.0}\PY{p}{,} \PY{p}{(}\PY{n}{width}\PY{p}{,} \PY{n}{height}\PY{p}{)}\PY{p}{)}
        
        \PY{c+c1}{\PYZsh{} Release the video in and open it again}
        \PY{n}{cap}\PY{o}{.}\PY{n}{release}\PY{p}{(}\PY{p}{)}
        \PY{n}{cap} \PY{o}{=} \PY{n}{cv2}\PY{o}{.}\PY{n}{VideoCapture}\PY{p}{(}\PY{l+s+s1}{\PYZsq{}}\PY{l+s+s1}{bubbles.mp4}\PY{l+s+s1}{\PYZsq{}}\PY{p}{)}
        
        \PY{n}{ret} \PY{o}{=} \PY{k+kc}{True}
        \PY{k}{while} \PY{p}{(}\PY{n}{ret} \PY{o+ow}{and} \PY{n}{cap}\PY{o}{.}\PY{n}{isOpened}\PY{p}{(}\PY{p}{)}\PY{p}{)}\PY{p}{:}
            \PY{n}{ret}\PY{p}{,} \PY{n}{frame} \PY{o}{=} \PY{n}{cap}\PY{o}{.}\PY{n}{read}\PY{p}{(}\PY{p}{)}
            \PY{k}{if} \PY{n}{ret} \PY{o}{==} \PY{k+kc}{True}\PY{p}{:}
        
                \PY{c+c1}{\PYZsh{} TODO \PYZhy{} add you filter here}
                \PY{n}{frame} \PY{o}{=} \PY{n}{cv2}\PY{o}{.}\PY{n}{flip}\PY{p}{(}\PY{n}{frame}\PY{p}{,} \PY{l+m+mi}{1}\PY{p}{)}
                \PY{c+c1}{\PYZsh{} Write the changed frame out}
                \PY{n}{out}\PY{o}{.}\PY{n}{write}\PY{p}{(}\PY{n}{frame}\PY{p}{)}
                
        \PY{c+c1}{\PYZsh{} When everything done, release the capture}
        \PY{n}{out}\PY{o}{.}\PY{n}{release}\PY{p}{(}\PY{p}{)}
        \PY{n}{cap}\PY{o}{.}\PY{n}{release}\PY{p}{(}\PY{p}{)}
        
        \PY{n+nb}{print}\PY{p}{(}\PY{l+s+s1}{\PYZsq{}}\PY{l+s+s1}{Finished}\PY{l+s+s1}{\PYZsq{}}\PY{p}{)}
\end{Verbatim}


    \begin{Verbatim}[commandchars=\\\{\}]
Finished

    \end{Verbatim}

    \subsubsection{3.2.1 - Link 1}\label{link-1}

    ORIGINAL VIDEO:

https://youtu.be/ABIeZHAgI8s

NEW VIDEO:

https://youtu.be/Dco-miCDsls

Question: Describe your filter and the results (single sentence answers
will not be accepted)

I wanted to see what the effect would be if I did a simple flip of the
video. I actually think the outcome is cool because it looks like it was
filmed with them standing the other direction the entire time. I think I
can see how easy it would be for someone to do a simple edit to make
something appear that it happened a specific way when it actually
didn't.

    \begin{Verbatim}[commandchars=\\\{\}]
{\color{incolor}In [{\color{incolor}4}]:} \PY{c+c1}{\PYZsh{} Open the sample video}
        \PY{n}{cap} \PY{o}{=} \PY{n}{cv2}\PY{o}{.}\PY{n}{VideoCapture}\PY{p}{(}\PY{l+s+s1}{\PYZsq{}}\PY{l+s+s1}{bubbles.mp4}\PY{l+s+s1}{\PYZsq{}}\PY{p}{)}
        
        \PY{c+c1}{\PYZsh{} Read the first frame and get the height, width and number of color channels}
        \PY{n}{ret}\PY{p}{,} \PY{n}{frame} \PY{o}{=} \PY{n}{cap}\PY{o}{.}\PY{n}{read}\PY{p}{(}\PY{p}{)}
        \PY{n}{height}\PY{p}{,} \PY{n}{width}\PY{p}{,} \PY{n}{channels} \PY{o}{=} \PY{n}{frame}\PY{o}{.}\PY{n}{shape}
        
        \PY{c+c1}{\PYZsh{} Output video must be the same size as the input in this example}
        \PY{n}{fourcc} \PY{o}{=} \PY{n}{cv2}\PY{o}{.}\PY{n}{VideoWriter\PYZus{}fourcc}\PY{p}{(}\PY{o}{*}\PY{l+s+s1}{\PYZsq{}}\PY{l+s+s1}{mp4v}\PY{l+s+s1}{\PYZsq{}}\PY{p}{)} \PY{c+c1}{\PYZsh{} Be sure to use lower case}
        \PY{n}{out} \PY{o}{=} \PY{n}{cv2}\PY{o}{.}\PY{n}{VideoWriter}\PY{p}{(}\PY{l+s+s1}{\PYZsq{}}\PY{l+s+s1}{myoutput2.mp4}\PY{l+s+s1}{\PYZsq{}}\PY{p}{,} \PY{n}{fourcc}\PY{p}{,} \PY{l+m+mf}{25.0}\PY{p}{,} \PY{p}{(}\PY{n}{width}\PY{p}{,} \PY{n}{height}\PY{p}{)}\PY{p}{)}
        
        \PY{c+c1}{\PYZsh{} Release the video in and open it again}
        \PY{n}{cap}\PY{o}{.}\PY{n}{release}\PY{p}{(}\PY{p}{)}
        \PY{n}{cap} \PY{o}{=} \PY{n}{cv2}\PY{o}{.}\PY{n}{VideoCapture}\PY{p}{(}\PY{l+s+s1}{\PYZsq{}}\PY{l+s+s1}{bubbles.mp4}\PY{l+s+s1}{\PYZsq{}}\PY{p}{)}
        
        \PY{n}{ret} \PY{o}{=} \PY{k+kc}{True}
        \PY{k}{while} \PY{p}{(}\PY{n}{ret} \PY{o+ow}{and} \PY{n}{cap}\PY{o}{.}\PY{n}{isOpened}\PY{p}{(}\PY{p}{)}\PY{p}{)}\PY{p}{:}
            \PY{n}{ret}\PY{p}{,} \PY{n}{frame} \PY{o}{=} \PY{n}{cap}\PY{o}{.}\PY{n}{read}\PY{p}{(}\PY{p}{)}
            \PY{k}{if} \PY{n}{ret} \PY{o}{==} \PY{k+kc}{True}\PY{p}{:}
        
                \PY{c+c1}{\PYZsh{} TODO \PYZhy{} add you filter here}
                \PY{n}{frame} \PY{o}{=} \PY{n}{cv2}\PY{o}{.}\PY{n}{flip}\PY{p}{(}\PY{n}{frame}\PY{p}{,} \PY{l+m+mi}{0}\PY{p}{)}
                \PY{c+c1}{\PYZsh{} Write the changed frame out}
                \PY{n}{out}\PY{o}{.}\PY{n}{write}\PY{p}{(}\PY{n}{frame}\PY{p}{)}
                
        \PY{c+c1}{\PYZsh{} When everything done, release the capture}
        \PY{n}{out}\PY{o}{.}\PY{n}{release}\PY{p}{(}\PY{p}{)}
        \PY{n}{cap}\PY{o}{.}\PY{n}{release}\PY{p}{(}\PY{p}{)}
        
        \PY{n+nb}{print}\PY{p}{(}\PY{l+s+s1}{\PYZsq{}}\PY{l+s+s1}{Finished}\PY{l+s+s1}{\PYZsq{}}\PY{p}{)}
\end{Verbatim}


    \begin{Verbatim}[commandchars=\\\{\}]
Finished

    \end{Verbatim}

    \subsubsection{3.2.2 - Link 2}\label{link-2}

    NEW VIDEO:

https://youtu.be/ebPzXJcgrsM

Question: Describe your filter and the results (single sentence answers
will not be accepted)

This time I decided to flip the video upside down. I decided to do this
even though I flipped the previous one horrizontily because I wanted to
see what kind of change it would have on the video. THis makes it look
like someone simply held the phone upside down while they were
recording. I think this is great because it shows that if I ever did
this with a video I would just need to flip it with cv to get it to be
in the correct direction

    \begin{Verbatim}[commandchars=\\\{\}]
{\color{incolor}In [{\color{incolor}5}]:} \PY{c+c1}{\PYZsh{} Open the sample video}
        \PY{n}{cap} \PY{o}{=} \PY{n}{cv2}\PY{o}{.}\PY{n}{VideoCapture}\PY{p}{(}\PY{l+s+s1}{\PYZsq{}}\PY{l+s+s1}{bubbles.mp4}\PY{l+s+s1}{\PYZsq{}}\PY{p}{)}
        
        \PY{c+c1}{\PYZsh{} Read the first frame and get the height, width and number of color channels}
        \PY{n}{ret}\PY{p}{,} \PY{n}{frame} \PY{o}{=} \PY{n}{cap}\PY{o}{.}\PY{n}{read}\PY{p}{(}\PY{p}{)}
        \PY{n}{height}\PY{p}{,} \PY{n}{width}\PY{p}{,} \PY{n}{channels} \PY{o}{=} \PY{n}{frame}\PY{o}{.}\PY{n}{shape}
        
        \PY{c+c1}{\PYZsh{} Output video must be the same size as the input in this example}
        \PY{n}{fourcc} \PY{o}{=} \PY{n}{cv2}\PY{o}{.}\PY{n}{VideoWriter\PYZus{}fourcc}\PY{p}{(}\PY{o}{*}\PY{l+s+s1}{\PYZsq{}}\PY{l+s+s1}{mp4v}\PY{l+s+s1}{\PYZsq{}}\PY{p}{)} \PY{c+c1}{\PYZsh{} Be sure to use lower case}
        \PY{n}{out} \PY{o}{=} \PY{n}{cv2}\PY{o}{.}\PY{n}{VideoWriter}\PY{p}{(}\PY{l+s+s1}{\PYZsq{}}\PY{l+s+s1}{myoutput4.mp4}\PY{l+s+s1}{\PYZsq{}}\PY{p}{,} \PY{n}{fourcc}\PY{p}{,} \PY{l+m+mf}{25.0}\PY{p}{,} \PY{p}{(}\PY{n}{width}\PY{p}{,} \PY{n}{height}\PY{p}{)}\PY{p}{)}
        
        \PY{c+c1}{\PYZsh{} Release the video in and open it again}
        \PY{n}{cap}\PY{o}{.}\PY{n}{release}\PY{p}{(}\PY{p}{)}
        \PY{n}{cap} \PY{o}{=} \PY{n}{cv2}\PY{o}{.}\PY{n}{VideoCapture}\PY{p}{(}\PY{l+s+s1}{\PYZsq{}}\PY{l+s+s1}{bubbles.mp4}\PY{l+s+s1}{\PYZsq{}}\PY{p}{)}
        
        \PY{n}{ret} \PY{o}{=} \PY{k+kc}{True}
        \PY{k}{while} \PY{p}{(}\PY{n}{ret} \PY{o+ow}{and} \PY{n}{cap}\PY{o}{.}\PY{n}{isOpened}\PY{p}{(}\PY{p}{)}\PY{p}{)}\PY{p}{:}
            \PY{n}{ret}\PY{p}{,} \PY{n}{frame} \PY{o}{=} \PY{n}{cap}\PY{o}{.}\PY{n}{read}\PY{p}{(}\PY{p}{)}
            \PY{k}{if} \PY{n}{ret} \PY{o}{==} \PY{k+kc}{True}\PY{p}{:}
        
                \PY{c+c1}{\PYZsh{} TODO \PYZhy{} add you filter here}
                \PY{n}{frameFlipped} \PY{o}{=} \PY{n}{cv2}\PY{o}{.}\PY{n}{flip}\PY{p}{(}\PY{n}{frame}\PY{p}{,} \PY{l+m+mi}{0}\PY{p}{)}
                \PY{n}{frameFlipped2} \PY{o}{=} \PY{n}{cv2}\PY{o}{.}\PY{n}{flip}\PY{p}{(}\PY{n}{frame}\PY{p}{,} \PY{l+m+mi}{1}\PY{p}{)}
                \PY{n}{frame} \PY{o}{=} \PY{n}{frameFlipped} \PY{o}{+} \PY{n}{frameFlipped2}
                \PY{c+c1}{\PYZsh{} Write the changed frame out}
                \PY{n}{out}\PY{o}{.}\PY{n}{write}\PY{p}{(}\PY{n}{frame}\PY{p}{)}
                
        \PY{c+c1}{\PYZsh{} When everything done, release the capture}
        \PY{n}{out}\PY{o}{.}\PY{n}{release}\PY{p}{(}\PY{p}{)}
        \PY{n}{cap}\PY{o}{.}\PY{n}{release}\PY{p}{(}\PY{p}{)}
        
        \PY{n+nb}{print}\PY{p}{(}\PY{l+s+s1}{\PYZsq{}}\PY{l+s+s1}{Finished}\PY{l+s+s1}{\PYZsq{}}\PY{p}{)}
\end{Verbatim}


    \begin{Verbatim}[commandchars=\\\{\}]
Finished

    \end{Verbatim}

    \subsubsection{3.2.2 - Link 3}\label{link-3}

    NEW VIDEO:

https://youtu.be/8Z7oETvJa1o

Question: Describe your filter and the results (single sentence answers
will not be accepted)

I think this is by far my favorite one. I ended up flipping both of my
previous two videos. Then I added those together and recorded those
videos so that they played together the results I think are amazing.

    \begin{center}\rule{0.5\linewidth}{\linethickness}\end{center}

\subsection{4 - Rubric}\label{rubric}

It is your responsibility to show that your assignment clearly satisfies
the rubric. Use as many tests to show that your assignment deserves the
grade it should get.

\begin{longtable}[]{@{}ll@{}}
\toprule
Task & Points\tabularnewline
\midrule
\endhead
Part 1 - applying filter to a video & 30\tabularnewline
Part 2 - video 1 & 20\tabularnewline
Part 2 - video 2 & 20\tabularnewline
Part 2 - video 3 & 20\tabularnewline
Coding style and Presentation & 10\tabularnewline
-\/-\/-\/-\/-\/- & -\/-\/-\/-\/-\tabularnewline
Total & 100\tabularnewline
\bottomrule
\end{longtable}

    \begin{center}\rule{0.5\linewidth}{\linethickness}\end{center}

\subsection{5 - Submission}\label{submission}

Do all of you coding in this Jupyter Notebook: 1. Download this notebook
to your computer. 1. Run Jupyter Notebook to allow you edit this
notebook. 1. Write and compile your code in the notebook. 1. When
finished, run all cells, export to HTML and upload it to I-Learn. 1.
\textbf{TEST YOUR LINKS}. The best method for testing your links is to
send your html file that you will be submitting to a friend to test. Or
at least, open your html file in the browser in "incognito" mode and
test your links.


    % Add a bibliography block to the postdoc
    
    
    
    \end{document}
